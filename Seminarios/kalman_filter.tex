% Options for packages loaded elsewhere
\PassOptionsToPackage{unicode}{hyperref}
\PassOptionsToPackage{hyphens}{url}
\PassOptionsToPackage{dvipsnames,svgnames,x11names}{xcolor}
%
\documentclass[
  letterpaper,
  DIV=11,
  numbers=noendperiod]{scrartcl}

\usepackage{amsmath,amssymb}
\usepackage{iftex}
\ifPDFTeX
  \usepackage[T1]{fontenc}
  \usepackage[utf8]{inputenc}
  \usepackage{textcomp} % provide euro and other symbols
\else % if luatex or xetex
  \usepackage{unicode-math}
  \defaultfontfeatures{Scale=MatchLowercase}
  \defaultfontfeatures[\rmfamily]{Ligatures=TeX,Scale=1}
\fi
\usepackage{lmodern}
\ifPDFTeX\else  
    % xetex/luatex font selection
\fi
% Use upquote if available, for straight quotes in verbatim environments
\IfFileExists{upquote.sty}{\usepackage{upquote}}{}
\IfFileExists{microtype.sty}{% use microtype if available
  \usepackage[]{microtype}
  \UseMicrotypeSet[protrusion]{basicmath} % disable protrusion for tt fonts
}{}
\makeatletter
\@ifundefined{KOMAClassName}{% if non-KOMA class
  \IfFileExists{parskip.sty}{%
    \usepackage{parskip}
  }{% else
    \setlength{\parindent}{0pt}
    \setlength{\parskip}{6pt plus 2pt minus 1pt}}
}{% if KOMA class
  \KOMAoptions{parskip=half}}
\makeatother
\usepackage{xcolor}
\setlength{\emergencystretch}{3em} % prevent overfull lines
\setcounter{secnumdepth}{-\maxdimen} % remove section numbering
% Make \paragraph and \subparagraph free-standing
\ifx\paragraph\undefined\else
  \let\oldparagraph\paragraph
  \renewcommand{\paragraph}[1]{\oldparagraph{#1}\mbox{}}
\fi
\ifx\subparagraph\undefined\else
  \let\oldsubparagraph\subparagraph
  \renewcommand{\subparagraph}[1]{\oldsubparagraph{#1}\mbox{}}
\fi


\providecommand{\tightlist}{%
  \setlength{\itemsep}{0pt}\setlength{\parskip}{0pt}}\usepackage{longtable,booktabs,array}
\usepackage{calc} % for calculating minipage widths
% Correct order of tables after \paragraph or \subparagraph
\usepackage{etoolbox}
\makeatletter
\patchcmd\longtable{\par}{\if@noskipsec\mbox{}\fi\par}{}{}
\makeatother
% Allow footnotes in longtable head/foot
\IfFileExists{footnotehyper.sty}{\usepackage{footnotehyper}}{\usepackage{footnote}}
\makesavenoteenv{longtable}
\usepackage{graphicx}
\makeatletter
\def\maxwidth{\ifdim\Gin@nat@width>\linewidth\linewidth\else\Gin@nat@width\fi}
\def\maxheight{\ifdim\Gin@nat@height>\textheight\textheight\else\Gin@nat@height\fi}
\makeatother
% Scale images if necessary, so that they will not overflow the page
% margins by default, and it is still possible to overwrite the defaults
% using explicit options in \includegraphics[width, height, ...]{}
\setkeys{Gin}{width=\maxwidth,height=\maxheight,keepaspectratio}
% Set default figure placement to htbp
\makeatletter
\def\fps@figure{htbp}
\makeatother

\usepackage{graphicx}
\usepackage{amsmath}
\KOMAoption{captions}{tableheading}
\makeatletter
\@ifpackageloaded{caption}{}{\usepackage{caption}}
\AtBeginDocument{%
\ifdefined\contentsname
  \renewcommand*\contentsname{Table of contents}
\else
  \newcommand\contentsname{Table of contents}
\fi
\ifdefined\listfigurename
  \renewcommand*\listfigurename{List of Figures}
\else
  \newcommand\listfigurename{List of Figures}
\fi
\ifdefined\listtablename
  \renewcommand*\listtablename{List of Tables}
\else
  \newcommand\listtablename{List of Tables}
\fi
\ifdefined\figurename
  \renewcommand*\figurename{Figure}
\else
  \newcommand\figurename{Figure}
\fi
\ifdefined\tablename
  \renewcommand*\tablename{Table}
\else
  \newcommand\tablename{Table}
\fi
}
\@ifpackageloaded{float}{}{\usepackage{float}}
\floatstyle{ruled}
\@ifundefined{c@chapter}{\newfloat{codelisting}{h}{lop}}{\newfloat{codelisting}{h}{lop}[chapter]}
\floatname{codelisting}{Listing}
\newcommand*\listoflistings{\listof{codelisting}{List of Listings}}
\makeatother
\makeatletter
\makeatother
\makeatletter
\@ifpackageloaded{caption}{}{\usepackage{caption}}
\@ifpackageloaded{subcaption}{}{\usepackage{subcaption}}
\makeatother
\ifLuaTeX
  \usepackage{selnolig}  % disable illegal ligatures
\fi
\usepackage{bookmark}

\IfFileExists{xurl.sty}{\usepackage{xurl}}{} % add URL line breaks if available
\urlstyle{same} % disable monospaced font for URLs
\hypersetup{
  pdftitle={Kalman Filter},
  pdfauthor={Felipe Marques},
  colorlinks=true,
  linkcolor={blue},
  filecolor={Maroon},
  citecolor={Blue},
  urlcolor={Blue},
  pdfcreator={LaTeX via pandoc}}

\title{Kalman Filter}
\author{Felipe Marques}
\date{}

\begin{document}
\maketitle

\subsection{Sumário}\label{sumuxe1rio}

\begin{enumerate}
\def\labelenumi{\arabic{enumi}.}
\tightlist
\item
  \hyperref[sec-introducao]{Introdução}
\item
  \hyperref[sec-state-space]{\emph{State Space Representation}}
\item
  \hyperref[sec-kalman-recursion]{\emph{Kalman Recursion}}

  \begin{enumerate}
  \def\labelenumii{\arabic{enumii}.}
  \tightlist
  \item
    \hyperref[sec-kalman-filter]{\emph{Kalman Filter}}
  \item
    \hyperref[sec-kalman-prediction]{\emph{Kalman Prediction}}
  \item
    \hyperref[sec-kalman-smoothing]{\emph{Kalman Smoothing}}
  \end{enumerate}
\item
  \hyperref[sec-estimacao]{Estimação dos parâmetros}
\item
  \hyperref[sec-inferencia]{Inferência com \emph{Kalman Filter}}
\end{enumerate}

\section{State Space Representation}\label{state-space-representation}

\begin{center}\rule{0.5\linewidth}{0.5pt}\end{center}

\subsection{Introdução}\label{sec-introducao}

A representação em estado de espaço é uma forma conveniente de
representar as dinâmicas de uma variável através de duas equações que a
descreve. (melhorar essa parte)

Essa representação será útil quando tratarmos dos algoritmos de Kalman.

\begin{center}\rule{0.5\linewidth}{0.5pt}\end{center}

\subsection{State Space Representation}\label{sec-state-space}

Considere,

\begin{itemize}
\tightlist
\item
  \(\underset{(n \times 1)}{\mathbf{y}_t}\) um vetor de retornos
  observados,
\item
  \(\underset{(r \times 1)}{\xi_t}\) um vetor possívelmente não
  observado (\emph{state vector}),
\item
  \(\underset{(k \times 1)}{\mathbf{x}_t}\) um vetor de variáveis
  exógenas,
\item
  \(\mathbf{F}_{(r \times r)}, \mathbf{A}_{(n \times k)}', \mathbf{H}'_{(n \times r)}\)
  matrizes de coeficientes.
\end{itemize}

\begin{center}\rule{0.5\linewidth}{0.5pt}\end{center}

Então, a representação em estado de espaço de um modelo é dada por:

. . .

\[
\begin{equation}
\begin{array}{ccl}
  \mathbf{y}_t & = & \mathbf{A}'\mathbf{x}_t + \mathbf{H}'\xi_t + \mathbf{w}_t \\
  \xi_{t+1} & = & \mathbf{F}\xi_t + \mathbf{v}_t
\end{array}
\end{equation}
\]

. . .

Onde, \(\mathbf{v}_t,\mathbf{w}_t \sim WN\)

. . .

\(\textrm{Var}(\mathbf{v}_t) = \mathbf{Q}\) e
\(\textrm{Var}(\mathbf{w}_t) = \mathbf{R}\). Além disso,
\(\textrm{Cov}(\mathbf{v}_t,\mathbf{w}_k) = 0, \forall t,k\).

. . .

A equação de cima é chamada de \emph{observation equation}, enquanto a
de baixo é chamada de \emph{state equation}.

\begin{center}\rule{0.5\linewidth}{0.5pt}\end{center}

\subsection{Exemplos}\label{exemplos}

. . .

Considere o processo AR(p) dado da forma:

. . .

\[
y_{t+1} - \mu = \phi_1(y_t - \mu) + \ldots + \phi_p(y_{t-p+1} - \mu) + \epsilon_{t+1}
\]

. . .

onde \(\epsilon_t \sim WN(0, \sigma^2)\)

Podemos escrever esse modelo na forma de espaço de estado como segue:

\begin{center}\rule{0.5\linewidth}{0.5pt}\end{center}

(\emph{Observation Equation})

\[
\scalebox{0.5}{
\begin{bmatrix}
  y_{t+1} - \mu \\
  y_t - \mu \\
  \vdots \\
  y_{t-p+2} - \mu
\end{bmatrix} = 
\begin{bmatrix}
  \phi_1 & \phi_2 & \ldots & \phi_{p-1} & \phi_p \\
  1 & 0 & \ldots & 0 & 0 \\
  0 & 1 & \ldots & 0 & 0 \\ 
  \vdots & \vdots & \ldots & \vdots & \vdots \\
  0 & 0 & \ldots & 1 & 0 \\
\end{bmatrix}
\begin{bmatrix}
  y_t - \mu \\
  y_{t-1} - \mu \\
  \vdots \\
  y_{t-p+1} - \mu
\end{bmatrix} +
\begin{bmatrix}
  \epsilon_{t+1} \\
  0 \\
  \vdots \\
  0
\end{bmatrix}}
\]

(\emph{State Equation})

\[
y_t = \mu + \begin{bmatrix}1 & 0 & \ldots & 0\end{bmatrix}\begin{bmatrix}
y_t - \mu \\
y_{t-1} \mu \\
\vdots \\
y_{t-p+1} - \mu
\end{bmatrix}
\]

\begin{center}\rule{0.5\linewidth}{0.5pt}\end{center}

\section{Kalman Recursion}\label{sec-kalman-recursion}

\begin{center}\rule{0.5\linewidth}{0.5pt}\end{center}

\subsection{Kalman Recursion}\label{kalman-recursion}

A recursão de Kalman é um método iterativo e recursivo para encontrar o
melhor estimador linear de \(\xi_t\), utilizando toda informação
disponível até o momento.

Nesse sentido, em geral, o interesse está em um desses 3 casos:

\begin{itemize}
\item
  \emph{filter} (\(\hat\xi_{t|t}\))
\item
  \emph{prediction} (\(\hat\xi_{t+1|t}\))
\item ~
  \subsection{\texorpdfstring{\emph{smoothing}
  (\(\hat\xi_{\tau|t}, \tau < t\))}{smoothing (\textbackslash hat\textbackslash xi\_\{\textbackslash tau\textbar t\}, \textbackslash tau \textless{} t)}}\label{smoothing-hatxi_taut-tau-t}
\end{itemize}

\subsection{Kalman Filter}\label{sec-kalman-filter}

\begin{center}\rule{0.5\linewidth}{0.5pt}\end{center}

\subsection{Kalman Prediction}\label{sec-kalman-prediction}

\begin{center}\rule{0.5\linewidth}{0.5pt}\end{center}

\subsection{Kalman Smoothing}\label{sec-kalman-smoothing}

\section{Estimação dos parâmetros}\label{sec-estimacao}

\subsection{Máxima verossimilhança}\label{muxe1xima-verossimilhanuxe7a}

A estimação dos parâmetros desconhecidos é feita através do método de
máxima verossimilhança.

Caso \(\xi_1\) e \(\{\mathbf{w}_t,\mathbf{v}_t\}\)

\section{Inferência com Kalman Filter}\label{sec-inferencia}



\end{document}
